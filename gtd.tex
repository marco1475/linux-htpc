\documentclass{article}
\usepackage[utf8]{inputenc}
\usepackage{graphicx}

\title{David Allen: \textit{Getting Things Done} (2nd ed.)}
\author{Marek Vojtko}
\date{July 2017}

\begin{document}

\maketitle

\section{The Art of Getting Things Done}

\subsection{A New Practice for a New Reality}

\begin{itemize}
  \item Clarify\marginpar{3} and organize all things that command your attention.
  \item Objectives:\marginpar{4}
  \begin{enumerate}
    \item Capture all things that need to be done in a \emph{logical} and trusted system \emph{outside} your head.
    \item Front-end decisions about all \emph{inputs} to generate a workable list of next actions.
    \item Curate and coordinate the list of next actions---what level\emph{s} of commitment do you have with yourself and others at any given time?
  \end{enumerate}
\end{itemize}

\paragraph{The Problem: New Demands, Insufficient Resources}

\subparagraph{Work No Longer Has Clear Boundaries}

\begin{itemize}
  \item Work\marginpar{5} lost ``edges'' and the ability to be considered ``done.''
  \item Everything could be better and you have access to the information to make it so---the internet.
  \item No\marginpar{6} edges also means more work: cross-team, -project, and -department.
  \item No edges also generate more communication (Skype with parents, etc.)
\end{itemize}

\subparagraph{Our Jobs (and Lives) Keep Changing}

\begin{itemize}
  \item Change\marginpar{7} comes from:
  \begin{enumerate}
      \item Morphing organizations.
      \item Being less employee, more a free agent.
      \item Increasing speed of change.
  \end{enumerate}
  \item No\marginpar{8} luxury of ``cruise control'' for extended periods of time, because of the increasing frequency of change.
\end{itemize}

\subparagraph{The Old Models and Habits Are Insufficient}

\begin{itemize}
  \item Post-industrial\marginpar{9} work:
  \begin{enumerate}
      \item Choice about what to do created the need for ABC organization, to-do lists, etc.
      \item Discretion about when to do it created the need for personal calendars.
  \end{enumerate}
\end{itemize}

\subparagraph{The Big Picture vs. the Nitty-Gritty}

\begin{itemize}
  \item ``Big-picture thinking'' fails because:
  \begin{enumerate}
      \item Too much distraction.
      \item Sub-conscious resistance to large projects.
      \item The more you see the more needs to change.
  \end{enumerate}
  \item Primary outcomes and values provide guidance on what to \emph{stop} doing and what to prioritize, but also increase the pressure of day-to-day.
  \item The\marginpar{11} real work does not happen at this high level; if you think of high-level goals, the nitty-gritty doesn't go away or becomes easier.
\end{itemize}

\paragraph{The Promise: The ``Ready State'' of the Martial Artist}

\begin{itemize}
  \item What is desired for optimal performance is being in the zone.
\end{itemize}

\subparagraph{The ``Mind Like Water'' Simile}

\begin{itemize}
  \item React\marginpar{12} like water to a pebble: appropriately, not too much or too little.
  \item Power in karate comes from speed, not muscle (the ``pop'' at the end of whip); a focused thrust with speed.
\end{itemize}

\paragraph{The Principle: Dealing Effectively with Internal Commitments}

\begin{itemize}
  \item Stress\marginpar{13} comes from inappropriately-managed commitments, the open loops (incompletes).
  \item Anything\marginpar{14} that doesn't belong where it is the way it is is an open loop, pulling at your attention.
  \item Closing the loop entails capturing it, clarifying what it means, and deciding how to move on it.
\end{itemize}

\subparagraph{The Basic Requirements for Managing Commitments}

\begin{enumerate}
    \item If it's on your mind, your mind isn't clear. \textbf{Capture it in a collection tool outside your mind that you regularly sort through.}
    \item Clarify commitments and decide what you have to do to make progress toward fulfilling it. \textbf{Break it down into actions.}
    \item Keep reminders of actions in a regularly-reviewed system.
\end{enumerate}

\subparagraph{An Important Exercise to Test This Model}

\begin{itemize}
  \item Clearer\marginpar{16} outcome definition and clear next action added control to the engagement with the world.
  \item We mostly thing \emph{of} a problem, not \emph{about} a problem.
  \item Structure your thinking toward an \textbf{outcome} and an \textbf{action}.
  \item Reacting is automatic, thinking is not.
\end{itemize}

\subparagraph{The Real Work of Knowledge Work}

\begin{itemize}
  \item ``In knowledge work\ldots the task is not given; it has to be determined. `What are the expected results from this work?' is\ldots the key question in making knowledge workers productive.''--Peter Drucker
\end{itemize}

\subparagraph{Why Things Are on Your Mind}

\begin{itemize}
  \item Things\marginpar{17} on your mind are wishes of change without:
  \begin{itemize}
    \item a clarified desired outcome,
    \item explicit next physical steps, and
    \item reminders added to a trusted system.
  \end{itemize}
\end{itemize}

\subparagraph{Your Mind Doesn't Have a Mind of Its Own}

\begin{itemize}
  \item Your\marginpar{18} mind will remind you of things at the times when you can do nothing about them, thus causing stress.
  \item Thinking about something you aren't making any progress on is wasted energy and it adds to your anxiety about doing things you shouldn't be doing.
  \item Your mind can't help keeping track of all open loops, but not as a positive motivator but as a detractor from anything else you need to think about, diminishing your performance.
\end{itemize}

\subparagraph{The Transformation of ``Stuff''}

\begin{itemize}
  \item Most organizing systems fail, because they don't transform all open loops (a.k.a.\ stuff).
  \item Convert\marginpar{19} all stuff into a clear inventory of meaningful actions, projects, and usable information.
  \item A to-do list is just a list of stuff that creates stress:
  \begin{itemize}
    \item Its items aren't broken down into actionable pieces, which means you still need to make a decision about each item.
    \item If you don't have the time/energy to do it when looking at the list, you will feel overwhelmed.
    \item It becomes an ``amorphous blob of undoability.''
  \end{itemize}
  \item Gather everything all the time, think about it all at a specific time, and do it all whenever you can.
\end{itemize}

\paragraph{The Process: Managing Action}

\begin{itemize}
  \item Get\marginpar{20} in the habit of keeping nothing on your mind.
\end{itemize}

\subparagraph{Managing Action Is the Prime Challenge}

\begin{itemize}
  \item You need to manage your actions.
  \item It\marginpar{21} is difficult to manage actions you haven't identified or decided on.
  \item The actions themselves, in their proper context, usually just take a minute or two.
  \item Lack of time usually means lack of clarity and definition.
  \item Clarify things at the front-end, when they appear, not on the back-end, when trouble has developed.
  \item Define what ``done'' means ($=$ outcome) and what ``doing'' looks like ($=$ action).
\end{itemize}

\subparagraph{The Value of a Bottom-Up Approach}

\begin{itemize}
  \item Improve personal productivity from the bottom up.
  \item Day-to-day stress prevents you from focusing on the larger picture.
  \item You\marginpar{22} need to be confident that your low-level tools will be able to handle the resulting tasks of high-level thinking.
\end{itemize}

\subparagraph{Horizontal and Vertical Action Management}

\begin{itemize}
  \item You need horizontal and vertical control:
  \begin{itemize}
    \item the breadth of all your disparate tasks (horizontal), and
    \item the\marginpar{23} depth of all connected tasks (vertical).
  \end{itemize}
\end{itemize}

\subparagraph{The Major Change: Getting It All Out of Your Head}

\begin{itemize}
  \item Record all ``open loops'' (thoughts), because they are not complete and hamper your thinking.
  \item There is an inverse relationship between how much something is on your mind and how much it's getting done.
  \item Making\marginpar{24} a list changes how you engage with the world.
  \item Don't waste time thinking about the same thing twice.
  \item Anything in an undecided state your mind will keep working on.
  \item Your\marginpar{25} mind is a focusing tool, not a storage place---distractions ($=$ cache invalidation) slow down the processing.
  \item Your mind has no sense of time, so it thinks anything you ought to do you have to do \emph{now}.
\end{itemize}

\subsection{Getting Control of Your Life: The Five Steps of Mastering Workflow}

\begin{itemize}
  \item Getting\marginpar{27} organized and setting priorities happen as a result of these five steps:
  \begin{enumerate}
      \item capture,
      \item clarify,
      \item organize,
      \item reflect, and
      \item engage.
  \end{enumerate}
  \item This is horizontal control.
  \item Quality\marginpar{28} of your workflow management is only as good as the weakest link in the five-step chain.
  \item You need to integrate and support each step with consistent standards.
  \item Common\marginpar{29} mistakes:
  \begin{itemize}
    \item Not capturing everything in your head.
    \item Not clarifying (next) actions.
    \item Not organizing the clarified results.
    \item Not reflecting often or consistently enough.
  \end{itemize}
  \item Any of the above mistakes leads to wrong decisions on what to engage with and when, usually driven by latest and loudest inputs, that are based on hope, not trust.
  \item Ask yourself: ``When do I need to see what, in what form, to get it off my mind?''
  \item It is important to keep these stages separate.
  \item Without separation you are organizing a vague list of ``most important'' things in some order of priorities and sequences without setting any real actions to take---you are not closing any open loops.
\end{itemize}

\paragraph{Capture}

\begin{itemize}
  \item Capture\marginpar{30} everything you have to do or \emph{decide}.
\end{itemize}

\subparagraph{Getting 100 Percent of the ``Incompletes''}

\begin{itemize}
  \item An incomplete is anything that ought to be different than it is and you want to change.
  \item Gather placeholders/representations of all incompletes in your world.
  \item Decisions\marginpar{31} whether or not to do something are incompletes as well.
  \item Capture incompletes in containers that you regularly empty.
  \item Anything that isn't completely captured in a trusted external tool remains resident in your mental space.
\end{itemize}

\subparagraph{The Capture Tools}

\begin{itemize}
  \item Use\marginpar{32} physical in-tray for physical (paper) things.
  \item Use digital devices and e-mail for digital things.
\end{itemize}

\subparagraph{The Success Factors for Capturing}

\begin{enumerate}
    \item Every\marginpar{33} open loop must be in your capture system and out of your head.
    \begin{itemize}
      \item If your mental ``in-tray'' is full, you won't be motivated to empty the physical ones.
      \item Two incomplete sets of things to do do not provide real payoffs for cleaning them out.
      \item Keep the collection tools with you at all times; nothing possibly useful should get lost.
    \end{itemize}
    \item You must have as few capturing buckets as you can get by with.
    \begin{itemize}
      \item The\marginpar{34} smaller the number of in-trays, the easier and more consistently you can process them.
      \item The digital world makes it especially difficult to keep to just a single in-tray (i.e. multiple e-mail addresses, calendars, or digital organizers).
    \end{itemize}
    \item You must empty the capture buckets regularly.
    \begin{itemize}
      \item Emptying\marginpar{35} in-trays does not mean finishing what is in them.
      \item You just process and organize them into your system; nothing ever goes back into the in-tray.
      \item To process means to clarify what it is, what should be done with it, and decide on the next action.
      \item Without an effective life-management system ``downstream,'' the in-trays will overflow and nothing will change.
      \item If you can't do anything about a task in the in-tray, do not leave it there!
      \item Once any pile gets too big, all informational value is lost.
    \end{itemize}
\end{enumerate}

\paragraph{Clarify}

\begin{itemize}
  \item Clarifying in-tray items requires item-by-item thinking.
  \item ``It\marginpar{36} is better to be wrong than vague.''--Freeman Dyson
  \item Personal organization is knowing what to ask yourself (and answer) about every single thing in your in-tray.
  \item Trying to organize things as they arrive will fail, because they arrive in incomplete batches and new things are constantly incoming.
  \item All you can do is capture and process them and you organize the actions you decided you needed to take based on the things in your in-tray.
  \item Refer to figure \ref{fig:gtd_workflow} for an overview of the clarifying steps\marginpar{37}.
  \item Note\marginpar{38} that for each actionable item you need to determine both, the project it belongs to and its next action.
\end{itemize}

\begin{figure}[htp]
  \centering
  \includegraphics[scale=0.6824]{gtd_workflow}
  \caption{The \textit{Getting Things Done} workflow.}
  \label{fig:gtd_workflow}
\end{figure}

\paragraph{Organize}

\begin{itemize}
  \item Eight\marginpar{39} categories of reminders:
  \begin{enumerate}
    \item trash (from non-actionable items),
    \item incubation (from non-actionable items),
    \item general reference files (from non-actionable items),
    \item list of projects (from actionable items),
    \item project-specific reference files (from actionable items),
    \item calendar (from actionable items),
    \item list of reminders for next actions (from actionable items), and
    \item list of reminders for things you are waiting for (from actionable items).
  \end{enumerate}
  \item A list is a reviewable set of reminders.
\end{itemize}

\subparagraph{Projects}

\begin{itemize}
  \item A\marginpar{41} project is any desired result that can be accomplished within a year that requires more than one action.
  \item You need to keep track of what is left to do through action reminders and weekly reviews.
  \item The Projects List does not need an order; it will be reviewed regularly enough to ensure appropriate next actions have been defined.
  \item You don't do projects, you do actions related to a project---projects are finish lines.
  \item Project-support material:
  \begin{itemize}
    \item Projects List is an index.
    \item Project materials should be separate from the Projects List.
    \item Unlike general reference material (from non-actionable items) the project-specific reference files need to be reviewed/purged more consistently to not miss any next actions.
  \end{itemize}
\end{itemize}

\subparagraph{The Next-Action Categories}

\begin{itemize}
  \item Next\marginpar{43} action is the next physical, visible behavior on every open loop.
  \item Next actions that\ldots
  \begin{itemize}
    \item \ldots need to happen on a specific date belong in the Calendar.
    \item \ldots need to happen as soon as possible belong on the Next Actions List.
    \item \ldots are blocked by something or someone belong on the Waiting For List.
  \end{itemize}
  \item Calendar:
  \begin{itemize}
    \item Contains only actions that need to happen on a specific day or at a specific time.
    \item Add time-specific actions (appointments), day-specific actions (to be done on a specific day, but not time, e.g. ``call me Friday''), and day-specific information\marginpar{44} (when something needs to be started or is due).
    \item Daily to-do lists in calendars don't work:
    \begin{itemize}
      \item New input and shifting priorities change the to-dos too fre\-quen\-tly---your game plans must be able to be renegotiated at any moment.
      \item Re-entering not-done items for some other day is demoralizing.
      \item Anything\marginpar{45} that absolutely doesn't have to be done on that day dilutes the emphasis on things that do.
    \end{itemize}
    \item Should be sacred territory; if you put something on it that thing has to get done done that day or not at all.
    \item Nothing wrong with ``if you have time'' lists as long as they don't get confused with ``have to do'' lists.
  \end{itemize}
  \item The ``Next Actions'' List
  \begin{itemize}
    \item Contains any action that will take longer than 2 minutes and cannot be delegated.
    \item Presents options for what we will do at any point in time.
    \item If it gets too long, feel free to subdivide into categories.
  \end{itemize}
\end{itemize}

\subparagraph{Non-actionable Items}

\begin{itemize}
  \item Trash\marginpar{46} is anything without potential future actions or reference value.
  \item Incubation:
  \begin{itemize}
    \item Use either a Someday / Maybe Lists or a Tickler File.
    \item Someday / Maybe Lists:
    \begin{itemize}
      \item ``Parking lot'' of ideas you'd like to get reminded about regularly.
      \item Other\marginpar{47} lists can be more specific: books to read, movies to watch, etc.
      \item Must be part of the Weekly Review.
    \end{itemize}
    \item Tickler File:
    \begin{itemize}
      \item Store ideas you don't need to reminded about until some designated time in the future.
      \item Calendar\marginpar{48} can serve this purpose.
    \end{itemize}
  \end{itemize}
  \item Reference Material:
  \begin{itemize}
    \item Anything intrinsically valuable as information.
    \item Must be \emph{easily} referenced/accessed.
    \item Two forms:
    \begin{enumerate}
        \item Topic- and area-specific storage (information defined by how it is stored).
        \item General reference files (ad-hoc information without a category).
    \end{enumerate}
    \item Must have ``nice, clean edges,'' otherwise actionable and non-actionable items will blur.
  \end{itemize}
\end{itemize}

\paragraph{Reflect}

\begin{itemize}
  \item Step\marginpar{49} back to review all needed work and dive into actionable items; rinse and repeat.
  \item Consistent use of the reflection step on a weekly basis is paramount.
\end{itemize}

\subparagraph{What to Review When}

\begin{itemize}
  \item Calendar\marginpar{50} will be reviewed most frequently.
  \begin{itemize}
    \item Determine the ``hard landscape'' of the day.
    \item See things that have to get done that day.
    \item Check what else needs to be done after you've finished a task.
  \end{itemize}
  \item The Next Actions List will be the next most-frequent.
  \begin{itemize}
    \item See the inventory of predefined actions that you can do whenever you have time during the day.
    \item Organize the list by context (\textit{At Home}, \textit{At Work}, \textit{Calls}, etc.) for faster filtering.
  \end{itemize}
  \item The Projects, Waiting For, and Someday/Maybe Lists need to be reviewed only to get them off your mind, but at least once a week.
\end{itemize}

\subparagraph{Critical Success Factor: The Weekly Review}

\begin{itemize}
  \item Prevent your mind from re-acquiring open loops by frequent-enough review of anything actionable.
  \item Retrench previous rapid and intuitive judgments at an elevated level.
  \item Review\marginpar{51} everything once a week to clear your mind and capture recent loose strands.
  \item Review steps:
  \begin{itemize}
    \item Gather and process all your stuff.
    \item Review your system.
    \item Update your lists.
    \item Get clean, clear, current, and complete.
  \end{itemize}
  \item Total overview is important, you cannot be left with a vague sense that something is missing.
\end{itemize}

\paragraph{Engage}

\begin{itemize}
  \item Facilitate\marginpar{52} good choices about what you are doing at any point in time.
  \item Move from hoping that you are doing the right thing to trusting that you are.
\end{itemize}

\subparagraph{Three Models for Making Action Choices}

\begin{itemize}
  \item Galvanize intuitive judgment with intelligent and practical thinking about the work.
  \item Three models help frame your options more intelligently:
  \begin{enumerate}
      \item The Four-Criteria Model for Choosing Actions in the Moment
      \begin{enumerate}
        \item \textbf{Context}\marginpar{53}, determined by requirements (e.g. location or tools) or capability.
        \item \textbf{Time available}, i.e.\ when do you have to do something else?
        \item \textbf{Energy available}, i.e.\ how tired are you?
        \item \textbf{Priority}, i.e.\ the highest payoff given the previous three filters.
      \end{enumerate}
      \item The Threefold Model for Identifying Daily Work
      \begin{enumerate}
        \item \textbf{Doing\marginpar{54} predefined work}, i.e.\ working off of the Next Actions List and Calendar, on tasks that were previously determined to need to get done.
        \item \textbf{Doing work as it shows up}, i.e.\ unsuspected, unforeseen things that you have/choose to engage with---you are deciding they are more important that the predefined work.
        \item \textbf{Defining your work}, i.e.\ breaking down projects into next actions, clearing your in-tray, etc. and taking care of less-than-two-minute tasks to achieve trust that your lists of things are current and complete.
      \end{enumerate}
      \item The Six-Level Model for Reviewing Your Own Work
      \begin{itemize}
        \item Priorities cannot be reliably determined---to know your priorities you need to know your work.
        \item Priorities\marginpar{55} depend on distance of perception (horizon):
        \begin{enumerate}
            \item \textbf{Ground: Current actions}, i.e.\ the accumulated list of all actions that you need to take.
            \item \textbf{Horizon 1: Current Projects}, i.e.\ projects that generate the actions at the Ground level, relatively short-term outcomes.
            \item \textbf{Horizon 2: Areas of Focus and Accountabilities}, i.e.\ your roles, responsibilities, and interests that create the pro\-jects at Horizon 1\marginpar{56}.
            \item \textbf{Horizon 3: Goals}, i.e.\ what you want to be experiencing in 2--3 years.
            \item \textbf{Horizon 4: Vision}, i.e.\ projecting 3--5 years into the future.
            \item \textbf{Horizon 5: Purpose and Principles}, i.e.\ big-picture view, ``Why?''; all goals, visions, objectives, projects, and actions derive from and lead to this.
        \end{enumerate}
        \item The levels might not fit perfectly, but they provide a useful framework for multi-layered thinking.
        \item Long-term\marginpar{57} goals are not a practical guide for day-to-day minutia, but managing the day-to-day allows you to think at the higher levels.
      \end{itemize}
  \end{enumerate}
\end{itemize}

\subsection{Getting Projects Creatively Under Way: The Five Phases of Project Planning}

\begin{itemize}
  \item Key\marginpar{58} ingredient of relaxed control is horizontal focus, consisting of:
  \begin{enumerate}
      \item clearly defined outcomes (projects) and the next actions needed to move them forward, and
      \item reminders placed in a trusted system that is reviewed regularly.
  \end{enumerate}
\end{itemize}

\paragraph{Enhancing Vertical Focus}

\begin{itemize}
  \item Vertical focus is evaluating all project-related things, mostly informally (``back of the napkin'' thinking).
  \item You\marginpar{59} don't need formal models for project thinking (if you do, you already use them); you need a project-focusing model, a way of validating and supporting your informal thinking.
  \item Format project planning (i.e. project management software) skips the ``why?'', doesn't allow for brainstorming, and isn't rigorous enough in determining next steps and accountabilities.
  \item Informal planning relieves stress because it clears open loops off the mind.
\end{itemize}

\paragraph{The Natural Planning Model}

\begin{itemize}
  \item Your\marginpar{60} brain goes through these stages for everything you do:
  \begin{enumerate}
      \item \textbf{Define purpose and principles}, the \textit{why}, i.e.\ the purpose is your intention and the principles are your boundaries.
      \item \textbf{Outcome visioning}, the \textit{what}, i.e.\ imagining what and where.
      \item \textbf{Brainstorming}, the \textit{how}, i.e.\ resolving the cognitive dissonance between now and the imagined.
      \item \textbf{Organizing}, i.e.\ sorting into sub-projects/components, priorities, and sequences of events by challenging, comparing, and evaluating.
      \item \textbf{Identifying\marginpar{62} next actions}, i.e.\ what to actually do next.
  \end{enumerate}
\end{itemize}

\subparagraph{Natural Planning Is Not Necessarily Normal}

\begin{itemize}
  \item Natural planning happens in your head, but not necessarily at work.
\end{itemize}

\paragraph{The Unnatural Planning Model}

\subparagraph{When the ``Good'' Idea Is a Bad Idea}

\begin{itemize}
  \item You\marginpar{63} can't have good ideas before the purpose is clear, the vision is well-defined, and data has been collected and organized.
\end{itemize}

\paragraph{The Reactive Planning Model}

\begin{itemize}
  \item The\marginpar{64} unnatural planning model is irrelevant to actual work and is therefore used less.
  \item This creates the reactive planning model (a.k.a.\ last-minute planning), which is the natural planning in reverse: action is needed, when that stalls you try to get organized, when you can't you brainstorm, until you hire a consultant who asks\marginpar{65} about vision and purpose as the first thing.
\end{itemize}

\paragraph{Natural Planning Techniques: The Five Phases}

\begin{itemize}
  \item Thinking about projects in natural ways makes things happen sooner, better, and more successfully.
\end{itemize}

\subparagraph{Purpose}

\begin{itemize}
  \item The impetus, the \textit{why}.
  \item Know and be clear about the purpose of any activity.
  \item Don't get caught up in details, in creating things.
  \item Thinking\marginpar{66} about why\ldots
  \begin{itemize}
    \item \ldots \textbf{defines success}: you can't win unless you know what winning is.
    \item \ldots \textbf{creates\marginpar{67} decision-making criteria} by asking if the task is worth the investment.
    \item \ldots \textbf{aligns resources} by asking if they are necessary.
    \item \ldots \textbf{motivates} by bringing the end into focus.
    \item \ldots \textbf{clarifies focus} by sorting things into important and expendable.
    \item \ldots \textbf{expands options} by\marginpar{68} expanding thinking about how to make the desired result happen.
  \end{itemize}
  \item Must be clear and specific: ``How will you know if this is off-purpose?'' must have a clear answer.
\end{itemize}

\subparagraph{Principles}

\begin{itemize}
  \item The monitoring.
  \item Are subconscious, but their violation leads to distraction and stress.
  \item ``I would give others totally free reign as long as\ldots'' leads you to your principles.
  \item ``What\marginpar{69} behavior might undermine what I'm doing and how can I prevent it?'' leads you to your standards.
  \item Define parameters of action and criteria for excellence of conduct.
\end{itemize}

\subparagraph{Vision/Outcome}

\begin{itemize}
  \item The blueprint, the \textit{what}.
  \item Have a clear picture of what success looks, sounds, and feels like.
  \item The power of focus:
  \begin{itemize}
    \item Ensure the highest level of unconscious support.
    \item The\marginpar{70} reticular activating system makes you unconsciously aware of information---the brain's ``search function'' is informed by what you focus on (e.g. an optometrist pays attention to people's glasses while a building contractor to beams and struts in the same room).
  \end{itemize}
  \item Clarifying\marginpar{71} outcomes:
  \begin{itemize}
    \item You won't see how to do something until you see yourself doing it.
    \item Successes in new and foreign territory are difficult to imagine without knowing exactly how to get there---that is backwards of how our mind work.
    \item Create clear outcomes first, but that requires:
    \begin{itemize}
      \item constantly (re)defining what you are trying to accomplish, and
      \item consistently reallocating resources towards completing tasks.
    \end{itemize}
    \item Outcome/vision\marginpar{72} can be a simple sentence of a full-on ``movie script'' with incredible detail.
  \end{itemize}
\end{itemize}

\subparagraph{Brainstorming}

\begin{itemize}
  \item The difference between ``is'' and ``vision'' gets automatically filled in.
  \item But the filler is random and of varying quality.
  \item Capture and express all ideas and evaluate/verify them later.
  \item Capturing your ideas:
  \begin{itemize}
    \item Captured\marginpar{73} ideas won't reoccur.
    \item Mind-mapping means brainstorming ideas into a graphic format.
  \end{itemize}
  \item Distributed\marginpar{74} cognition:
  \begin{itemize}
    \item External brainstorming helps generate new ideas, because the captured ideas are reflected back at you.
    \item Means getting things out of your mind and into objective, reviewable formats, i.e. building an ``external mind.''
    \item Use ``cognitive artifacts'' (e.g. pen, paper, etc.) as an anchor for your ideas; without them your focus will slip.
  \end{itemize}
  \item Brainstorming\marginpar{75} keys:
  \begin{itemize}
    \item Don't judge, challenge, evaluate, or criticize.
    \begin{itemize}
      \item This is the unnatural planning model.
      \item Avoid self-censorship and trying to say the ``right'' thing.
      \item Critical thoughts are okay, but the primary goal must be inclusion and expansion, not constriction and contraction.
    \end{itemize}
    \item Go for quantity, not quality.
    \begin{itemize}
      \item Quantity helps your thinking in being expansive.
      \item Big-store\marginpar{76} shopping: the more choices (thoughts) the more confidence you will have in your decision.
    \end{itemize}
    \item Put analysis and organization in the background.
    \begin{itemize}
      \item Don't organize your thoughts prematurely.
    \end{itemize}
  \end{itemize}
\end{itemize}

\subparagraph{Organizing}

\begin{itemize}
  \item Means identifying components, sequences, and/or priorities.
  \item If you thoroughly emptied your mind during brainstorming, a natural organization emerges---your brain is always looking for patterns.
  \item Use structuring tools (i.e.\ project management tools).
  \item Also\marginpar{77} fosters creativity---seeing things organized might prompt new items to be added.
  \item The basics of organizing:
  \begin{itemize}
    \item Identify significant pieces.
    \item Sort by (one or more): components, sequences, priorities.
    \item Detail to the required degree.
  \end{itemize}
  \item Different projects need different amounts of structure and detail.
\end{itemize}

\subparagraph{Next Actions}

\begin{itemize}
  \item Are decisions about allocations of physical resources.
  \item Grounded, reality-based thinking and clarification desired outcomes defines and clarifies what ``real work'' is.
  \item If\marginpar{78} you are not ready to answer ``What is the next action?'' you have more fleshing out to do at one of the previous stages.
  \item The basics:
  \begin{itemize}
    \item Decide on the next action:
    \begin{itemize}
      \item for each ``moving part'' of each project, and
      \item in the planning process.
    \end{itemize}
    \item A sufficiently-planned project has a next action settled for every front that can be moved forward without any dependencies that need to be fulfilled first.
    \item ``Is there something somebody could be doing on this right now?"
    \item Clarifying next actions, including next planning actions, is crucial for relaxed control.
    \item If\marginpar{79} the next action is someone else's it still must be clarified (and put on the Waiting For List).
    \item Next-action conversation forces organizational clarity---resource allocation needs tend to do that.
  \end{itemize}
\end{itemize}

\subparagraph{How Much Planning Do You Really Need to Do?}

\begin{itemize}
  \item Things remain on your mind because:
  \begin{itemize}
    \item outcomes and actions have not been appropriately defined, or
    \item reminders have not been setup properly, i.e.\ in places where you can trust you will look at them.
  \end{itemize}
  \item 80\% of projects need planning in your head only.
  \item 15\% of projects need some brainstorming (e.g. mind-maps, write-ups, etc.).
  \item 5\% \marginpar{80} of projects need deliberate application of the five stages of the natural planning model.
  \item Need more clarity?
  \begin{itemize}
    \item Move up the the planning process, toward lower numbers.
  \end{itemize}
  \item Need more to be happening?
  \begin{itemize}
    \item Move down the planning process, toward higher numbers.
  \end{itemize}
\end{itemize}

\paragraph{}

\begin{itemize}
  \item Actions\marginpar{81}: Collect open loops, apply front-end thought process to each of them, and manage results with organization, review, and action.
  \item Projects: Use the five-stage planning model to create actions that get you from here to there.
\end{itemize}

\section{Practicing Stress-Free Productivity}

\subsection{Getting Started: Setting Up the Time, Space, and Tools}

\begin{itemize}
  \item Full\marginpar{85} implementation will take two full days.
\end{itemize}

\paragraph{Implementation, Whether All-Out or Casual, Is a Lot About ``Tricks''}

\begin{itemize}
  \item Often tricks are all you need.
  \item Secret\marginpar{87} to efficiency is to put the right things in focus at the right time.
\end{itemize}

\paragraph{Setting Aside the Time}

\begin{itemize}
  \item Full capturing process can take 6 hours.
  \item Clarifying and deciding on action can take 8 hours.
  \item After\marginpar{88} hours is an ideal time for tackling similar tasks of lower importance (``rabbit trails'').
\end{itemize}

\paragraph{Setting Up the Space}

\begin{itemize}
  \item If\marginpar{89} you have two work locations, make sure they have identical, mirrored systems.
  \item You need a discrete space dedicated to the processing of incoming information and open loops (work space $+$ in-tray).
\end{itemize}

\subparagraph{If You Go to an Office, You'll Still Need a Space at Home}

\begin{itemize}
  \item Have\marginpar{90} a satellite office at home.
  \item An office space in transit: a way to work in your system on the go to take advantage of unexpected downtime.
  \item Don't share space: Separate in-trays, or better, workstations, are key---do not share them with your significant other.\marginpar{91}
  \item Achieve zero resistance to using the system
\end{itemize}

\paragraph{Getting the Tools You Need}

\subparagraph{The Basic Processing Tools}

\begin{itemize}
  \item Set\marginpar{92} up a physical in-tray for mail, checks, prescriptions, etc. that will need to be evaluated later.
  \item One\marginpar{93} thought per piece of paper is very valuable, because it separates initial thoughts into discrete placeholders.
  \item Reduce friction to filing.
  \item Calendar\marginpar{94} isn't for action lists, it's for keeping track of the ``hard landscape;'' it also isn't the main organizing tool, but a part of a system.
  \item Once\marginpar{95} you know how to process your stuff and what to organize, you really just need to create and manage lists.
  \item Using\marginpar{96} tools you love to use greatly enhances productivity, but the tool will not provide you with stress-free productivity---the process does.
  \item All the tool wants to do is manage lists; create and review them easily and regularly.
\end{itemize}

\subparagraph{The Critical Factor of a Filing System}

\begin{itemize}
  \item Lack\marginpar{97} of a good general-reference filing system inordinately clouds your physical and mental space.
  \item Random, non-actionable, but potentially relevant material, when unprocessed and unorganized, produces psychological noise and blocks workflows, as other stuff backs up around it.
  \item Reference systems have to be fast, functional, and fun.
  \item Discrete filing systems cover contracts, finances, etc. (you already have this).
  \item A general-reference filing system holds articles, notes, etc., i.e.\ anything that doesn't belong in the discrete filing system, but has interesting or useful data worth keeping.
  \item Success\marginpar{99} factors for filing:
  \begin{itemize}
    \item You need a streamlined (\textless 1 minute) system for both, paper-based and digital general-reference filing.
    \item The system needs to be fast, fun, and easy, current and complete, otherwise you'll subconsciously resist filing things and they will accumulate in inappropriate places.
    \item Filing something new and adding to an existing item must be equally easy and fun.
    \item Keep\marginpar{100} the reference files immediately at hand.
    \item Alphabetical filing is better than organizing by category.
    \item Do not use your files as a personal management system because once you forget where you filed something, the options magnify geometrically.
    \item Digital tags can provide organization, but also generate friction.
    \item Avoid\marginpar{101} digital ``write-only'' syndrome---you need to sort and use the information intelligently, not just record it.
    \item Categorization is better than search.
    \item Make it easy to create a new folder.
    \item Have plenty of space---physical drawers should always be \textless \( \frac{3}{4} \) full.
    \item If\marginpar{102} the information is worth keeping, it is worth keeping close at hand; if the information isn't worth keeping close at hand, why are you keeping it at all?
    \item Label file folders with an auto-labeler---things you name, you own, collected but unnamed things own you.
    \item Purge\marginpar{103} your files at least once a year.
  \end{itemize}
  \item Filing as a Success Factor Itself:
  \begin{itemize}
    \item Filing isn't low priority, because it unclutters mental and physical workspaces.
    \item Items\marginpar{104} of different types in the same pile make the brain go numb towards the pile.
    \item Reference material cannot ``bleed'' into other categories.
  \end{itemize}
\end{itemize}

\subsection{Capturing: Corralling Your Stuff}

\begin{itemize}
  \item Capturing\marginpar{106} everything is the first critical steps toward ``mind like water,'' i.e. being on top of things.
  \item You have to trust that you are working with the whole picture of your world in your system.
  \item You won't always capture everything 100\% right when it happens, but you have to get to it later.
\end{itemize}

\paragraph{Ready, Set\ldots}

\begin{itemize}
  \item Capturing\marginpar{107} everything:
  \begin{itemize}
    \item provides a sense of volume of stuff,
    \item  shows you the ``end of the tunnel,'' and
    \item avoids distractions (``something is somewhere'') during clarifying and organizing.
  \end{itemize}
  \item You are capturing all the things that don't belong where they are in one location.
  \item By definition, these things aren't that important---they haven't blown up (yet) into importance, but you also haven't dealt with them or decided to drop them entirely.
  \item But because there \emph{could} be something important the things are controlling you and take up more energy than they deserve.
  \item You can only feel good about what you are not doing when you know everything that you are not doing.
\end{itemize}

\paragraph{\ldots Go!}

\subparagraph{Physical Gathering}

\begin{itemize}
  \item Anything\marginpar{108} that isn't in its correct place goes into the in-tray, as quickly as possible.
  \item These things stay where they are:
  \begin{itemize}
    \item Supplies (post-its, stationary, etc.).
    \item Reference material (should be in general or project-specific filing).
    \item Decorations\marginpar{109} (pictures, plants, etc.).
    \item Equipment (phone, computer, etc.).
  \end{itemize}
  \item \ldots unless they have an action tied to them.
  \item Issues with capturing:
  \begin{itemize}
    \item If\marginpar{110} the item is too big to fit in the in-tray, write a note to represent it and date the note---get into the habit of date-stamping everything you write (by hand or digitally).
    \item If the pile is too big to fit in the in-tray, create overflow stacks around the in-tray as needed.
    \item Dump anything immediately obvious as trash, but don't get distracted by the decision---if you are unsure about a thing, it goes in the in-tray and will be decided on later (clarifying requires a different mindset than capturing).
    \item Beware\marginpar{111} the ``purge-and-organize'' bug---it is better to capture things to do as future projects/actions than to do them immediately.
    \item Don't get distracted; the goal is to set up the system fully as soon as possible.
    \item Any existing lists should be treated as new items that need to be processed (again).
    \item Anything\marginpar{112} crucial that gets discovered is either emergent-enough to deal with immediately or it goes in the in-tray.
  \end{itemize}
  \item Start with your desktop:
  \begin{itemize}
    \item Most people use their desktop as their in-tray, so start here.
    \item Don't leave anything in a stack that you ``know what's there''---that's exactly what hasn't worked before.
    \item If anything about the desktop (equipment, etc.) needs changing, write a note a put it in your in-tray.
  \end{itemize}
  \item Do\marginpar{113--114} the same for desk drawers, countertops, cabinets, bulletin boards, shelves, equipments, etc.
  \item For a truly empty head it is imperative to do this everywhere you have stuff.
  \item This is not about minimalism:
  \begin{itemize}
    \item Anything\marginpar{115} that you are even slightly uncomfortable with throwing out should stay.
    \item But you need to capture and clarify any projects or actions embedded in the things you keep and put them where they belong.
  \end{itemize}
\end{itemize}

\subparagraph{Mental Gathering: The Mind Sweep}

\begin{itemize}
  \item Anything on your mind that needs doing.
  \item Each thought, idea, project, or thing gets its own placeholder (sheet of paper).
  \item Expect\marginpar{116} to spend 20 minutes to an hour.
  \item Things will occur to you randomly.
  \item Go for quantity, you will toss the junk later.
  \item Review the Incompletion Triggers\marginpar{116--120}.
\end{itemize}

\paragraph{The ``In'' Inventory}

\begin{itemize}
  \item Your\marginpar{121} in-tray should also contain voice mails, e-mails, etc.
  \item Capturing is complete when you can easily see the outer edges to the inventory of everything you need to do.
  \item Print out any tasks and to-do lists from other apps and put them in the in-tray.
\end{itemize}

\subparagraph{But ``In'' Doesn't Stay in ``In''}

\begin{itemize}
  \item Anything left in the in-tray will creep back into your mind, so you need to process and organize it all.
\end{itemize}

\subsection{Clarifying: Getting ``In'' to Empty}

\begin{itemize}
  \item Clarifying\marginpar{122} means getting to the bottom of your now overflowing in-tray.
  \item But not by doing the things, just by identifying each item and deciding what it is, what it means, and what you are going to do with it.
  \item Processing is easy, adding things to the organization system is harder.
\end{itemize}

\paragraph{Processing Guidelines}

\begin{itemize}
  \item Process\marginpar{124} the top item first.
  \item Process one time at a time.
  \item Never put anything back into ``in.''
\end{itemize}

\subparagraph{Top Item First}

\begin{itemize}
  \item Processing means to decide what the thing is, what action is required, and dispatching it accordingly.
  \item Emergency scanning is not processing:
  \begin{itemize}
    \item There\marginpar{125} are no favorites, or more or less important items.
    \item Playing favorites will invariably leave things unprocessed.
    \item You process from start to finish, no exceptions.
  \end{itemize}
\end{itemize}

\subparagraph{One Item at a Time}

\begin{itemize}
  \item Don't\marginpar{126} get distracted by easier decisions.
  \item Always focus on a single thing.
\end{itemize}

\subparagraph{Nothing Goes Back into ``In''}

\begin{itemize}
  \item Putting\marginpar{127} a thing back into the in-tray will have wasted a decision (and contributed to decision fatigue).
  \item Deciding not to decide is still a decision, but it's a bad one if you leave the thing unprocessed.
\end{itemize}

\paragraph{The Key Processing Question: ``What's the Next Action?''}

\begin{itemize}
  \item For each item in your in-tray make a firm next-action decision.
\end{itemize}

\subparagraph{What If There Is No Action?}

\begin{itemize}
  \item Trash\marginpar{128}:
  \begin{itemize}
    \item Processing things makes you aware of what not to do as much as of what to do; this will produce trash.
    \item Decide whether to keep or trash things you are in doubt about (minimalist vs. hoarder, which are you?).
    \item Too much information is the same as too little information: you don't have what you need, when and how you need it.
    \item Storing\marginpar{129} things digitally is easy, but they still have to be organized.
    \item Avoid constant input but no utilization by regularly reviewing and purging of outdated information and a more conscious filtering on the front-end.
  \end{itemize}
  \item Incubate:
  \begin{itemize}
    \item Nothing to do now, but there will be later.
    \item Add\marginpar{130} the items to either the Someday/Maybe List or put a reminder of them on your Calendar/Tickler File to get them off your mind immediately with the confidence that you will be reminded of them.
    \item It's okay to decide not to decide if you have a decide-not-to-decide system that catches these non-decisions---this is how you get things off your mind.
  \end{itemize}
  \item Reference:
  \begin{itemize}
    \item Potentially useful information about projects/topics.
    \item File it in your general-reference filing system.
    \item An\marginpar{131} under-1-minute general-reference filing system is mission critical.
    \item For anything you want to keep: make a label, put it in a file folder, and tuck it in your file drawer.
    \item Don't keep a ``to file'' pile---if it won't get filed immediately, it won't get filed.
    \item Keep e-mails in folders within the e-mail app, don't hesitate to create new folders and keep at inbox zero.
    \item Regular revisiting of data will keep any storage effective.
    \item Unpack any attention on your reference content or system.
  \end{itemize}
\end{itemize}

\subparagraph{And If There Is an Action\ldots What Is It?}

\begin{itemize}
  \item Decide\marginpar{132} what the next action is.
  \item Next action is the next physical, visible activity required to move the situation toward closure.
  \item Often quick analyses and several planning steps have to happen first.
  \item The\marginpar{133} action step needs to be the absolute next physical thing to do.
  \begin{itemize}
    \item ``Set meeting'' is not an action, because it doesn't describe a physical activity.
    \item Finish ``thinking exercises'' about each item.
    \item Without physical steps there will be a psychological gap every time you even vaguely think about the item, which will lead to procrastination.
    \item When\marginpar{134} it's time to do the action all the thinking has to have been done earlier. \item You can then more easily and effectively use the time and the tools to get things done, having already defined what there is to do.
    \item There is always a physical activity that can be done to facilitate decision making---use that activity rather than ``decide X'' as the next action.
    \item 99\% of the time you just need more information, which comes from outside (other people) or inside (thinking, drafting, etc.).
  \end{itemize}
\end{itemize}

\subparagraph{Once You Decide What the Next Action Step Is}

\begin{itemize}
  \item Do it:
  \begin{itemize}
    \item If it is an action that will take you less than 2 minutes.
    \item This includes even low-priority actions---if you are going to do it at all, do it now.
    \item At\marginpar{135} the two-minute mark it would take longer to file it appropriately than to do it.
    \item The two-minute rule is just a guideline; depending on how long you have to process your in-tray it can be 5 or 10 minutes.
    \item It\marginpar{136} might help to time yourself for a few of these actions, so you get a feeling for how long 2 minutes are.
    \item Some actions generate new actions that need to be tracked (do, delegate, or defer).
    \item The\marginpar{137} two-minute rule is for engaging with new input; don't spend all day doing two-minute tasks.
  \end{itemize}
  \item Delegate it:
  \begin{itemize}
    \item If the next action takes longer than 2 minutes, ask yourself whether you are the best person for it.
    \item Use a systematic format (e-mail, notes, text, voicemail, calendar, talk) to hand off the action---prefer written formats.
    \item Track\marginpar{138} the action and the handoff in a Waiting For List, if you care about the outcome.
    \item The same applies to actions that are already in someone else' ``court.''
    \item Record dates on everything you hand off so you can refer back to it.
  \end{itemize}
  \item Defer\marginpar{139} it:
  \begin{itemize}
    \item Most of the actions will be yours and will take longer than 2 minutes.
    \item These actions have to be organized into appropriate categories.
  \end{itemize}
\end{itemize}

\subparagraph{The Pending Things That Are Left}

\begin{itemize}
  \item Delegated and deferred actions are ``pending'' and need to be added to and tracked by your system.
\end{itemize}

\paragraph{Identifying the \textit{Projects} You Have}

\begin{itemize}
  \item After you have identified all next actions you need to shift your focus from single-action to larger-picture: projects.
  \item A project is any outcome you've committed to achieving that will take more than one action to complete.
  \item These\marginpar{140} are things you know will need more actions after the next action has been completed.
  \item Projects are the stakes in the ground that remind you of other actions required after the current one has been completed.
  \item Make a Projects List that serves as a placeholder for all your open loops.
  \item The Projects List must be maintained, as it is the key driver for reviewing week-to-week momentum.
\end{itemize}

\subsection{Organizing: Setting Up the Right Buckets}

\begin{itemize}
  \item The\marginpar{141} \textit{Getting Things Done} system declutters the mind, allowing it to intuitively focus, because unresolved matters have been dealt with.
  \item The physical organization system must be better than your mental one for that to happen.
  \item Something is organized when where it is matches what it means to you.
  \item You have to clarify things before you can know what they mean to you.
  \item You will create your organization system over time, not all at once---it will evolve as you process stuff and test out whether everything is in the best place for \emph{you}.
  \item What\marginpar{142} things mean to you will stay the same, but the organization structure will evolve.
\end{itemize}

\paragraph{The Basic Categories}

\begin{itemize}
  \item Projects List
  \item Project support material
  \item Calendar actions and information
  \item Next Actions List
  \item Waiting For List
  \item Reference material
  \item Someday/Maybe List
\end{itemize}

\subparagraph{The Importance of Hard Edges}

\begin{itemize}
  \item Each category must be kept visually, physically, and psychologically stric\-tly distinct because they represent different types of agreements with ourselves, with specific reminder times and ways.
  \item Capturing and clarifying your relationship to these things is therefore paramount.
  \item Just ``getting organized'' leads to rearranging incomplete inventories of unclear things.
  \item Losing\marginpar{144} the distinction between categories will make you go psychologically numb to their contents.
\end{itemize}

\subparagraph{All You Really Need Are Lists and Folders}

\begin{itemize}
  \item Most lists fail because they don't contain the right things or all of the things.
  \item Do\marginpar{145} not create external structuring of priorities on your lists, because you'll just have to rearrange and change them.
  \item The list is there just to keep track of the total inventory of active things to which you have made a commitment.
\end{itemize}

\paragraph{Organizing Action Reminders}

\begin{itemize}
  \item Deferred actions, reminders for delegated actions, calendar reminders, and someday/maybe actions need to be sorted into groups which you'll be able to review as options for work.
\end{itemize}

\subparagraph{The Actions That Go on Your Calendar}

\begin{itemize}
  \item Actions need to get done either as soon as possible (they still may have a specific final due date) or on a specific date (time- or day-specific).
  \item Do\marginpar{146} not put things (to-dos) that you would \emph{like} to do at a certain time into your Calendar.
  \item The Calendar is hard landscape only, i.e.\ the hard edges of your day's commitments need to be noticeable at a glance on the run.
  \item All other actions are ``are soon as possible, against all other things I have to do.''
\end{itemize}

\subparagraph{Organize As--Soon--As--Possible Actions by Context}

\begin{itemize}
  \item Context is a tool, location, or situation that allows you to complete the action, e.g.\ \textit{At Computer}, \textit{At Home}, \textit{Calls}, \textit{Errands}, etc.
  \item How\marginpar{147} discrete these categories need to be depends on how many actions you have to track and how often you change contexts.
  \item Once a list becomes too cluttered to see/review comfortably, subdivide---but avoid unproductive re-sorting.
  \item Context switches are costly---having prepared lists of all the things you need to do while in a context increases productivity.
  \item Reduce\marginpar{148} friction wherever you can, e.g.\ add phone numbers to each action in the \textit{Calls} context, have an \textit{Online} context separate from the \textit{At Computer} context, etc.
  \item Think about \emph{where}, \emph{when}, and under \emph{what circumstances} you can do which actions and organize your lists accordingly.
  \item Sub-contexts\marginpar{149} are great, e.g.\ a \textit{Hardware Store} list under the \textit{Errands} context.
  \item Location-based reminders, while at first glance great, hamper the overview of all inventory and lack context---they are only a nice add to a rigorously-managed and reviewed list.
  \item Simplifying your focus on actions will ensure that more of them get done.
  \item Use\marginpar{150--151} agendas for interacting with people (separate sub-lists for each person and (regular) meeting).
  \item Make a Read/Review\marginpar{152} List for things you want to read and maybe a Review/Respond List for more rigorous, focused material.
\end{itemize}

\subparagraph{Organizing ``Waiting For''}

\begin{itemize}
  \item Sort\marginpar{153} and group reminders of actions the results of which you are waiting on.
  \item Track final deliverables instead of discrete actions steps.
  \item Have a trigger for what you are waiting for and from whom.
  \item Manage the commitments of others before their avoidance creates a crisis.
  \item The Waiting For List should be close at hand, next to your Next Actions List in your system because the responsibility for an action might bounce back and forth many times---moving items between lists is an important feature for to-do software.
  \item It\marginpar{154} is very important that each item on the Waiting For List has a date stamp.
\end{itemize}

\subparagraph{Using the Original Item as Its Own Action Reminder}

\begin{itemize}
  \item Track action reminders by adding them to lists immediately as they occur.
  \item The originating trigger (notes, voicemails, etc.) won't be needed anymore after the action has been processed.
  \item Exceptions are certain kinds of input that server better themselves as reminders rather than an entry on a list, e.g.\ actionable e-mail and papers).
  \item Managing paper-based workflow:
  \begin{itemize}
    \item ``Read\marginpar{155} Vogue magazine'' is overkill when you can just have it in your physical reading-tray; the same goes for paper bills.
    \item However, any physical reminders must be visible enough to be re\-min\-ders---separate actionable papers/e-mails from the rest.
    \item Whether to add actions to a list or use the item itself depends on whether the action is limited to a single space (context), in which case use the item, or is doable somewhere else as well (use the reminder).
    \item Physical\marginpar{156} reminders should be visibly discrete categories based on their next action, not their type.
    \begin{itemize}
      \item The goal is to front-load all decision making to an earlier time and then go through stacks of work with a single action/purpose.
      \item If decisions are still pending, your mind will go numb.
    \end{itemize}
  \end{itemize}
  \item Managing e-mail-based workflow:
  \begin{itemize}
    \item Some e-mails are their own best reminders.
    \item Keeping e-mails as reminders in your inbox is okay, as long as you have less than 1 screen full of e-mails at any given time.
    \item Otherwise create 2 folders for e-mails that you need to deal with:
    \begin{enumerate}
        \item E-mails taking longer than 2 minutes to deal with (\texttt{@ACTION}).
        \item E-mails\marginpar{157} you are waiting for a reply on (\texttt{@WAITINGFOR}).
    \end{enumerate}
    \item Keeping inbox at zero then makes it function like a clue to process something.
    \begin{itemize}
      \item Staging undecided actionable things, reference, and trash in your inbox numbs the mind because you have to reassess everything every time.
      \item Inbox zero does not mean everything is done, it just means that everything is in its right place (i.e.\ organized), waiting to be acted upon.
    \end{itemize}
    \item You\marginpar{158} still need to regularly review the \texttt{@ACTION} folder and deal with the e-mails when you have time.
  \end{itemize}
  \item A caution about dispersing reminders of your actions:
  \begin{itemize}
    \item The primary function of an organization system is to provide reminders when and where you need to see them, so you can trust your choices about what you are (not) doing.
    \item Therefore you need to handle the \texttt{@ACTION} e-mail folder as an extension of your \textit{At Computer} context.
    \item Adding reminders anywhere (out of sight) is okay as long as you regularly review them.
  \end{itemize}
\end{itemize}

\paragraph{Organizing Project Reminders}

\subparagraph{The Project List(s)}

\begin{itemize}
  \item The\marginpar{159} Projects List is not meant to hold plans or details about the projects themselves, nor is it supposed to be arranged by priority or urgency---it's just a comprehensive list/index of your open loops.
  \item You won't be working off the Projects List in your moment-to-moment work; instead you'll be working with your Calendar, Next Action List, and any unexpected tasks that creep up.
  \item You can't \emph{do} a project, you can only do the actions it requires, but projects provide control over longer reaches of time (``Moving from tree-hugging to forest management.'')
  \item The Projects List provides, once weekly, a complete review of everything, making sure you have actions defines for everything and that everything is prioritized appropriately.
  \item The value of a complete Projects List:
  \begin{itemize}
    \item A project is more than one action requireed to achieve the desired result.
    \item An\marginpar{160} complete and current inventory of all projects\ldots
    \begin{itemize}
      \item \ldots is critical for control and focus, because it alleviates internal pressure due to open loops.
      \item \ldots alleviates subtle tensions---small, subtle things are often more difficult to handle and frequently produce projects of their own once they become bigger than expected.
      \item \ldots is\marginpar{161} the core of the weekly review, which marries longer commitments to day-to-day activities (the Projects List must exist before you can think from that perspective).
      \item \ldots facilitates relationship management, because it provides an overview of all commitments/open loops relating to a given person.
    \end{itemize}
  \end{itemize}
  \item Where to look for projects still to uncover:
  \begin{itemize}
    \item Current activities: Inventory your Calendar, Next Action Lists, and workspaces.
    \item Higher-horizon\marginpar{162} interests and commitments: Your professional and personal goals and responsibilities, ``look into'' projects.
    \item Current\marginpar{163} problems, issues, and opportunities:
    \begin{itemize}
      \item Problems are always projects, but the existence and achievability of a solution needs to be determined.
      \item Process improvements, i.e.\ anything that crosses the line between mildly irritating and a real bother.
      \item Creative\marginpar{164} and capacity-building opportunities could be in your Someday/Maybe List, but incorporating them in your Projects List by defining desired outcomes improves your life.
    \end{itemize}
  \end{itemize}
  \item One list is usually best, because it provides only placeholders for all open loops and that doesn't hamper intuitive moment-to-moment strategic decisions.
  \item Sub-sorting the Projects List is okay, as long as you review all sub-sections during the Weekly Review.
  \item Some common ways to sub-sort projects:
  \begin{itemize}
    \item Personal vs. professional.
    \item Delegated\marginpar{165} projects.
    \item Specific types of projects to allow different sorting.
  \end{itemize}
  \item Your organization will change as you will grow more familiar with the system.
  \item What about sub-projects?
  \begin{itemize}
    \item As\marginpar{166} long as you regularly review all sub-projects feel free to sub-divide projects as much as you'd like.
    \item A big project may require a list of sub-projects in project supporting materials if some of its sub-projects are interdependent or have strict priorities while others are independent---your next action can be moving from sub-project to sub-project, if one gets stuck.
    \item How\marginpar{167} you list projects and sub-projects doesn't really matter as long as you know where to find all the moving pieces.
  \end{itemize}
\end{itemize}

\subparagraph{Project Support Materials}

\begin{itemize}
  \item Project support materials \emph{aren't} actions or reminders.
  \item Using them as reminders means that next actions and waiting-for items have not been determined for them.
  \item Any unclarified reminder repels the mind instead of attracting it to the next action---they don't prompt you to do anything, they just add stress and anxiety.
  \item However,\marginpar{168} you may want to keep project support materials more readily accessible than general reference materials for easier access during the Weekly Review or while doing next actions.
  \item Ad-hoc\marginpar{168} project thoughts need to be organized at the time they occur to you as\ldots
  \begin{itemize}
    \item \ldots attached notes (in software) or Post-Its.
    \item \ldots e-mails\marginpar{170} in dedicated reference folders.
    \item \ldots paper-based files (your general-reference filing system\marginpar{171} must make you feel comfortable about creating a new folder for scraps of paper from a meeting).
    \item \ldots pages in notebooks.
  \end{itemize}
  \item Be sure to review \emph{all} of the methods you use consistently for next action steps.
  \item Also clear out any inactive, unreal or redundant notes so your projects support material doesn't grow stale.
  \item Digital usage lends itself easily to spreading information to many different places (apps, web pages, etc.), which can be counter-productive---you need to be able to see everything integrated from the right perspective at the right time.
\end{itemize}

\paragraph{Organizing Non-actionable Data}

\begin{itemize}
  \item Blending actionable things with non-actionable data and material is a common mistake.
  \item Managing\marginpar{172} non-actionable things is just as important as managing next actions and reminders.
  \item Non-actionable things fall into three categories: reference materials, reminders of things that need no action now but might at a later date, and things that you don't need at all (trash).
\end{itemize}

\subparagraph{Reference Materials}

\begin{itemize}
  \item No action required, but information you want to keep.
  \item Decide how much to keep, what form to store it in, and where to store it.
  \item You have to decide what is actionable and what isn't---if you don't it will still clutter your brain.
  \item Any reference material should have no action associated with it, no pull or incompletion.
  \item Adding pure reference material does not add to psychological weight.
  \item The variety of reference systems:
  \begin{itemize}
    \item General-reference\marginpar{173} filing:
    \begin{itemize}
      \item 1--4 physical drawers, many digital folders and categories.
      \item Must be easily accessible and usable.
    \end{itemize}
    \item Large-category filing:
    \begin{itemize}
      \item Any topic requiring 50$+$ folders or documents gets its own (physical) space, alphabetically sorted.
      \item Limit\marginpar{174} the amount of these to avoid uncertainty about where to file something.
    \end{itemize}
    \item Contact managers:
    \begin{itemize}
      \item Information directly related to people.
      \item Do not make your contacts manager a tool for reminding.
    \end{itemize}
    \item Libraries\marginpar{175} and archives:
    \begin{itemize}
      \item Group things according to frequency of use, location of use, importance, etc.
    \end{itemize}
  \end{itemize}
  \item First\marginpar{176} determine actionable vs. non-actionable, then determine the usage case, which informs how and where to store it.
  \item Reference systems are highly personal, but keep in mind the ratio of value received vs. time and effort required for capturing and maintaining.
  \item Craft a system from the ground up, practically, rather than top down, theoretically.
  \item Tolerate ambiguity and review and assess the system to dynamically cour\-se-correct.
\end{itemize}

\subparagraph{Someday/Maybes}

\begin{itemize}
  \item Backburner, non-actionable items that are on-hold.
  \item Someday/Maybe\marginpar{177} List:
  \begin{itemize}
    \item Writing things down makes things happen, because the power of imagination can foster changes in perception and performance.
    \item Permits you to imagine cool things without having to commit to doing anything.
    \item Your imagination and your current list of projects are sour\-ces: if\marginpar{178} something optional on your Projects List won't get worked on for months, put it on the Someday/Maybe List.
  \end{itemize}
  \item Special categories of someday/maybe:
  \begin{itemize}
    \item The difference\marginpar{179} between someday/maybe and reference is that someday/maybe gets regularly reviewed.
    \item Some lists (e.g.\ books to read, etc.) might be better off on the Someday/Maybe List than in reference so you'll be reminded of it.
  \end{itemize}
  \item Do not use ``hold and review'' files and piles, because the review never comes, which creates numbness and resistance toward the pile.
  \item Using\marginpar{180} the calendar for future options:
  \begin{itemize}
    \item Day-specific triggers for activating projects with lead-up time, i.e.\ when they happen you add a project to your active Projects List.
    \item Events\marginpar{181} you might want to attend---determine how far in advance you need to decide whether you're attending a put a trigger in your Calendar.
    \item Decision catalysts for decisions you can't (or don't want to) make now, but will be able to make in the future.
    \begin{itemize}
      \item Only as long as what you are waiting on is \emph{internal}; if it's external it goes on the Waiting For List.
      \item In order to be okay with not deciding you need a safety net (i.e.\ reminders) that will make you decide later.
    \end{itemize}
  \end{itemize}
  \item The ``tickler'' file:
  \begin{itemize}
    \item Also\marginpar{182} known as ``suspense,'' ``bring forward,'' ``perpetual'' or ``follow-up'' file.
    \item Organizes non-actionable items that may need an action in the future.
    \item You are ``mailing'' yourself things to receive at a certain date/time.
    \item Is a simple file folder system that allows you to distribute (physical) reminders such that whatever you want to see on a particular date in the future ``automatically'' shows up that day in your in-tray.
    \item It\marginpar{183} requires a one-second-per-day new behavior to make it work, but its payoff is much greater.
    \item Consists of a perpetual folder that contains files for the next 31 days and the next 12 months.
    \item A\marginpar{184} physical folder allows you to store the actual documents needed.
    \item You must check and update it every day and if you can't, you have to update the days you'll miss before you miss them.
  \end{itemize}
\end{itemize}

\paragraph{Checklists: Creative and Constructive Reminders}

\begin{itemize}
  \item Checklists\marginpar{185} are your external brain that allow for a more relaxed control.
  \item Thinking about anything with regularity should create a checklist.
\end{itemize}

\subparagraph{First, Identify Inherent Projects and Actions}

\begin{itemize}
  \item Fuzzy\marginpar{186} checklist items need to be clarified into projects and broken down into next actions.
  \item Some items can't be clarified further (e.g.\ ``Keep my team motivated''), but they still need to be tracked.
\end{itemize}

\subparagraph{Blueprinting Key Areas of Work and Accountability}

\begin{itemize}
  \item These\marginpar{187} fuzzy items translate to higher-level views of your ``work'', e.g.\ career goals, family, relationships, health and energy, finances, etc.
  \item Each of these levels might have reminders of responsibility---use these reminders to keep the ship on course, on an even keel.
  \item The\marginpar{188} more novel the situation, the higher the need for control, i.e.\ a checklist.
  \item Checklists can be useful in letting you know what you \emph{don't} need to be concerned about.
\end{itemize}

\subparagraph{Checklists at All Levels}

\begin{itemize}
  \item The\marginpar{189} more checklists, the better (e.g.\ travel, Weekly Review, year-end activities, etc.).
  \item Make\marginpar{190} it easy to create and edit checklists.
\end{itemize}

\subsection{Reflecting: Keeping It All Fresh and Functional}

\begin{itemize}
  \item Reflection\marginpar{191} engages your brain on a consistent basis with \emph{all} your commitments and activities.
  \item You must be assured that you are doing what you need to be doing and that it's okay to be \emph{not} doing what you are not doing---this makes you present in the moment.
  \item For your brain to trust the external system, the system has to current and complete.
\end{itemize}

\paragraph{What to Look At, When}

\begin{itemize}
  \item You\marginpar{192} need to see all the action options when you need to see them.
  \item The reviews happen in a few-second intervals all through-out the day: Got a phone and some spare time? Glance at the \textit{Calls} list and decide which call to make.
\end{itemize}

\subparagraph{Look at Your Calendar First\ldots}

\begin{itemize}
  \item To\marginpar{193} see the hard landscape and assess what needs to get done.
  \item Also look at your daily Tickler folder, if you're maintaining one.
\end{itemize}

\subparagraph{\ldots Then Your Action Lists}

\begin{itemize}
  \item To see what you can do in the current context.
  \item You'll evaluate the actions against all other work coming your way to ensure you are making the best decisions possible.
\end{itemize}

\subparagraph{Right Review in the Right Context}

\begin{itemize}
  \item The Calendar and Next Actions List are usually the only lists you will peruse this frequently.
  \item With all lists up-to-date you will be better prepared for any impromptu meeting, whether it be with your boss or with your spouse.
\end{itemize}

\paragraph{Updating Your System}

\begin{itemize}
  \item Trustworthiness\marginpar{194} of the whole system is enforced by regularly-refreshed thinking about and changes to the system from an elevated perspective.
  \item Your lists have to always be up-to-date, otherwise your brain will re-engage---enter the Weekly Review.
\end{itemize}

\subparagraph{The Power of the Weekly Review}

\begin{itemize}
  \item The\marginpar{195} Weekly Review builds in time into your busy schedule to capture, re-evaluate, and re-process all the stuff that's coming at you faster than you can handle.
  \item It also sharpens your focus on the most important projects.
  \item You will have to say ``no'' to more stuff faster and the Weekly Review's insistence on project-level thinking makes that easier.
  \item What is the Weekly Review?
  \begin{itemize}
    \item It is whatever you need to get your head empty again and get oriented for the next (couple of) weeks.
    \item It is going through the steps of workflow management (capture, clarify, organize, review) until you can say, ``I absolutely know right now everything I'm not doing, but could be doing if I decided to.''
    \item It is a three-part process:
    \begin{enumerate}
        \item Get clear, i.e.\ ensure all your collected stuff is processed.
        \begin{itemize}
          \item Collect\marginpar{196} everything you received from everywhere and put it in the in-tray.
          \item Get the in-tray to empty:
          \begin{itemize}
              \item Review all notes and messages.
              \item Decide on and list all action items, projects, waiting-fors, calendar events, and someday/maybes.
              \item File any reference materials.
              \item Get to inbox zero.
          \end{itemize}
          \item Empty your head:
          \begin{itemize}
            \item Put into writing any projects, action items, waiting-fors, and someday/maybes.
          \end{itemize}
        \end{itemize}
        \item Get clear, i.e.\ ensure all lists are up-to-date and reviewed.
        \begin{itemize}
          \item Review your Next Actions List:
          \begin{itemize}
            \item Mark off completed actions.
            \item Add any new actions for which you have reminders.
          \end{itemize}
          \item Review previous calendar data:
          \begin{itemize}
            \item Review the past 2 or 3 weeks of your calendar for remaining or emergent action items, reference, etc.
            \item Be\marginpar{197} able to archive your past Calendar with nothing left uncaptured.
          \end{itemize}
          \item Review upcoming calendar data:
          \begin{itemize}
            \item Long- and short-term entries.
            \item Capture actions about projects and preparation required for upcoming events.
          \end{itemize}
          \item Review your Waiting-For List:
          \begin{itemize}
            \item Follow-ups, pokes, etc.
            \item Record any new actions, check off ones already received.
          \end{itemize}
          \item Review your Projects List:
          \begin{itemize}
            \item Evaluate the status of each one.
            \item Make sure each project has at least one current kick-start action in your Next Actions List.
            \item Browse through the project support material.
          \end{itemize}
          \item Review any relevant checklists.
        \end{itemize}
        \item Get creative, i.e.\  capture automatically-generated ideas and perspectives that came up during the review process.
        \begin{itemize}
          \item Getting\marginpar{198} organized eliminates a barrier to the flow of creative energies.
          \item Get your act together, let spontaneous ideas emerge, capture them, and utilize their value.
          \item Review your Someday/Maybe List:
          \begin{itemize}
            \item Are there any projects to activate, delete, or add?
          \end{itemize}
          \item Be creative and courageous.
        \end{itemize}
    \end{enumerate}
    \item Make sure to do the Weekly Review well and consistently.
  \end{itemize}
  \item The right time and place for the review:
  \begin{itemize}
    \item Trick\marginpar{199} your brain into giving up a couple of hours a week for the Weekly Review.
    \item The ideal time are 2 hours early in the afternoon of your last working day---the week is still fresh in your memory, you can still reach people if you need to, and it makes the weekend more relaxing.
    \item Plane\marginpar{200} trips are also a good time.
    \item You need a weekly regrouping ritual, because your best thoughts about work won't happen at work.
    \item Don't\marginpar{201} neglect the higher-level (Horizon 1) review and exert self-guided integrated thinking.
  \end{itemize}
\end{itemize}

\paragraph{The ``Bigger Picture'' Reviews}

\begin{itemize}
  \item Bigger-picture reviews (Horizons 3--5) are implicitly guiding your decisions, so don't forget to revisit them.
  \item Keeping your everyday world under control will facilitate thinking about long-term goals.
  \item You\marginpar{202} need to assess your life and work at the appropriate horizons, making the appropriate decisions, at the appropriate intervals.
  \item Feeling confident in handling all you must already and thinking you can handle new things are prerequisites to evaluating long-term goals.
  \item Give at least two years to the \textit{Getting Things Done} system to get you to that point.
  \item This\marginpar{203} future thinking allows you to stay flexible and informal about goal setting---you are agile.
\end{itemize}

\subsection{Engaging: Making the Best Action Choices}

\begin{itemize}
  \item You\marginpar{204} must trust your intuition, but you can do things to enhance this trust.
  \item Context\marginpar{205} for deciding actions (as always, bottom up):
  \begin{itemize}
    \item The four-criteria model for choosing actions in the moment.
    \item The threefold model for evaluating daily work.
    \item The six-level model for reviewing your own work.
  \end{itemize}
\end{itemize}

\paragraph{The Four-Criteria Model for Choosing Actions in the Moment}

\begin{enumerate}
  \item Context
  \item Time available
  \item Energy available
  \item Priority
\end{enumerate}

\subparagraph{Context}

\begin{itemize}
  \item What can you possibly do where you are with the tools that you have?
  \item Organizing\marginpar{206} action reminders by context prevents needless reconsidering of all your actions.
  \item Do not create a Miscellaneous Actions List---the specificity of each list's context forces you to clarify the next action more accurately/thoroughly.
  \item Creative context sorting:
  \begin{itemize}
    \item Temporary\marginpar{207} contexts (e.g.\ ``before trip'') and fun contexts (e.g.\ ``brain gone'') are encouraged.
    \item Contexts are personal and will change over time, so make them be whatever works for you.
    \item Balance structure (ease of retrieval) with input time (how complicated it is to add things).
    \item Tags (with time estimates as well) might work.
  \end{itemize}
\end{itemize}

\subparagraph{Time Available}

\begin{itemize}
  \item Have\marginpar{208} a balance of long, large tasks and short, quick tasks to fill the weird, empty time in your day.
\end{itemize}

\subparagraph{Energy Available}

\begin{itemize}
 \item Match\marginpar{209} productive activity with your vitality level.
 \item Keep a list of tasks that require very little thinking to fill low-energy time slots.
 \item Clean edges in your management system are crucial:
 \begin{itemize}
  \item When you can't find simple, no-thinking items on your to-do list, because it is in disarray, you won't do anything because it will be too much work just to find something to work on.
  \item Your action lists have to be complete and current with all the thinking having been done already.
 \end{itemize}
\end{itemize}

\subparagraph{Priority}

\begin{itemize}
 \item If\marginpar{210} you are in the right context, have the time and the energy, which remaining tasks are the most important?
 \item Conscious decisions about accountabilities, goals, and values help you feel good about what you didn't get done---see the six horizons.
\end{itemize}

\paragraph{The Threefold Model for Evaluating Daily Work}

\begin{itemize}
 \item You need to set priorities relative to all other work (i.e.\ any commitment you have to make something happen) you can do right now.
 \item The three types of activities you'll be engaged during the day:
 \begin{enumerate}
   \item Doing\marginpar{211} predefined work.
   \item Doing work as it shows up.
   \item Defining your work.
 \end{enumerate}
 \item Most people get sucked into 2., neglecting 1.\ and 3.
 \item Urgent\marginpar{212} in-the-moment work is fine, as long as you know what you are neglecting and you review/re-evaluate everything later. (To know what you are neglecting you must process your in-tray and review your lists.)
 \item If you know all you have to do and are \emph{choosing} to handle the urgent in-the-moment works, you are being efficient.
 \item There are no interruptions, only mismanaged inputs.
 \item Don't let yourself get caught up in the urgency of the moment without feeling comfortable about what you are not dealing with.
 \item Without a well-functioning system you won't be able to pause (or bookmark):
 \begin{itemize}
  \item whatever you were working on, or
  \item any new action resulting from what you were working on, or
  \item the new work,
 \end{itemize}
 and be confident that you'll be able to come back to it or that it will be surfaced appropriately.
 \item Mismanaged\marginpar{213} action lists tend to produce emergencies from any item that's been around too long.
 \item It is easier to handle the emergent tasks than to define your work and manage your total work inventory; to avoid this, process input rapidly into a rigorously-defined system.
\end{itemize}

\subparagraph{The Moment-to-Moment Balancing Act}

\begin{itemize}
 \item Your\marginpar{214} gut will tell you how long you can prioritize emergent work over processing your in-tray while still feeling on top of things.
 \item Efficient dispatching (doing or processing) of incoming work and productive use of the ``weird time'' allows you to quickly and easily shift between tasks.
 \item Humans cannot multitask effectively; making your brain organize your work all the time forces you to multitask daily.
 \item You\marginpar{215} need to be aware of larger goals/contexts to know when to switch tasks according to your roles.
\end{itemize}

\paragraph{The Six-Level Model for Reviewing Your Own Work}

\begin{itemize}
 \item The six levels of work:
 \begin{itemize}
  \item Horizon 5: Life
  \item Horizon 4: Long-term visions
  \item Horizon 3: One- to two-year goals
  \item Horizon 2: Areas of focus and accountability
  \item Horizon 1: Current projects
  \item Ground: Current actions
 \end{itemize}
 \item Each level enhances and aligns with the one above it.
 \item Manage\marginpar{216} all the levels in a balanced fashion.
 \item Identify all open loops, incompletions, and commitments on \emph{all} levels.
\end{itemize}

\subparagraph{Working from the Bottom Up}

\begin{itemize}
 \item You\marginpar{217} can approach your priorities from any level at any time.
 \item The trick is to pay attention to the ones you need to at the appropriate time to keep you clear and present with whatever you are doing.
 \item Managing yourself from the top down without a sense of control at the implementation level is frustrating.
 \item The\marginpar{218} more lasting approach is to get control of the bottom level and then elevating your focus.
 \item Managing the bottom is universal and provides flexibility---it frees your mind to focus on higher-level things.
 \item The most important thing to deal with is whatever is \emph{most} on your mind, regardless of whether you want it on your mind or not.
 \item When you handle whatever has your attention you notice what \emph{really} has your attention; rinse and repeat.
 \item ``If\marginpar{219} your boat is sinking, you really don't care in which direction it's pointed!''
 \item Ground ($=$ actions):
 \begin{itemize}
  \item Ensure your action lists are complete.
  \item If you have \textless 50 actions, you haven't gotten all of them.
 \end{itemize}
 \item Horizon 1 ($=$ projects):
 \begin{itemize}
  \item Finalize your Projects List.
  \item Defines week-to-week operational world.
  \item Most\marginpar{220} challenging items to clarify/plan reside here; you must consistently identify objective outcomes and next actions.
 \end{itemize}
 \item Horizon 2 ($=$ areas of responsibility and personal commitments):
 \begin{itemize}
  \item Current job responsibilities (work ``hats'').
  \item Create\marginpar{221} ``areas of focus'' checklists (personal ones separate from professional ones) for self-management, which will serve as a trigger for new projects every 1--3 months.
  \item Your\marginpar{222} job description is a constantly-changing discussion that you need to stay on top of.
 \end{itemize}
 \item Horizon 3--5:
 \begin{itemize}
  \item Factors of the future and your direction and intention become pri\-mary---\textit{Ground}, \textit{Horizon 1} and \textit{2} are all about the current state.
  \item ``What\marginpar{223} is true right now about where I've decided I'm going and how am I going to get there?''
  \item 1-year goals $=$ \textit{Horizon 3}, 3-year vision $=$ \textit{Horizon 4}, your life's purpose $=$ \textit{Horizon 5}.
  \item Capture what motivators exist in your current state.
  \item This\marginpar{224} is less about coming up with new goals and more about discovering projects and actions that are already implied but unrealized.
 \end{itemize}
\end{itemize}

\subparagraph{Getting Priority Thinking Off Your Mind}

\begin{itemize}
 \item Write\marginpar{225} down and process anything that popped into your head while examining your higher-level goals.
\end{itemize}

\subsection{Getting Projects Under Control}

\begin{itemize}
 \item Up\marginpar{227} until now we dealt with horizontal thinking---now it's time to engage with individual items vertically.
\end{itemize}

\paragraph{The Need for More Informal Planning}

\begin{itemize}
 \item Do more planning, more informally, more often.
 \item Planning\marginpar{228} captures and utilizes the creative and proactive thinking we (could) do.
 \item Going top down runs into a lack of usable systems, because it results in potentially infinite amount of detail.
 \item The bottom up approach frees up room for constructive thinking, but you need to have systems and habits ready to leverage the ideas.
 \item Set up systems and tricks to get yourself to think about projects and situations more frequently, more easily, and more in-depth.
 \item Don't save your thinking for big, formal meetings or planning sessions.
\end{itemize}

\paragraph{What Projects Should You Be Planning?}

\begin{itemize}
 \item Most things on the Projects List won't require more than a few seconds of planning.
 \item Planning is required if\ldots
 \begin{itemize}
  \item \ldots the project has your attention even after you determined the next action.
  \begin{itemize}
   \item Apply\marginpar{229} the four phases of natural planning (purpose and principles, vision/outcome, brainstorming, and organizing).
  \end{itemize}
  \item \ldots ideas and details about the project show up ad-hoc.
  \begin{itemize}
   \item Store the thoughts in a place for later use/review.
  \end{itemize}
 \end{itemize}
\end{itemize}

\subparagraph{Projects That Need Next Actions About Planning}

\begin{itemize}
 \item Some projects need ``more planning'' as their next action.
 \item Add it as an action to the appropriate action list and then proceed with further planning.
 \item Typical planning steps:
 \begin{itemize}
  \item Brainstorming: If the next action is unclear, make ``draft ideas re:'' your next action.
  \item Organizing:\marginpar{230} If you need to organize the project, ``organize project'' works.
  \item Setting up meetings.
  \item Gathering information.
 \end{itemize}
\end{itemize}

\subparagraph{Random Project Thinking}

\begin{itemize}
 \item Any\marginpar{231} random project-related thoughts that aren't next actions need to be recorded, either in your in-tray (for later processing) or directly in project-specific reference material.
\end{itemize}

\paragraph{Tools and Structures That Support Project Thinking}

\subparagraph{Thinking Tools}

\begin{itemize}
 \item Thinking\marginpar{232} tools help you stay focused and stimulate thinking (function follows form).
\end{itemize}

\subparagraph{Writing Instruments}

\begin{itemize}
 \item Always have something with you to capture your thoughts.
 \item Smartphones\marginpar{234} are for the execution of the results of thinking, not necessarily for the thinking itself.
\end{itemize}

\subparagraph{The Support Structures}

\begin{itemize}
 \item A general-reference filing system is integral not just to managing the general workflow process, but also to project thinking.
 \item Create a folder for a topic as soon as you have something to put in it.
 \item Any scrap of information about any new project you need to parse for next actions, then create a new folder in your general-reference filing system and store it there.
 \item Despite\marginpar{235} digital, paper is still important---we tend to think differently based on how we can express ourselves.
 \item Paper is also a better reminder than a digital file.
 \item Use\marginpar{236} a digital mind-mapping tool to brainstorm and organize your projects.
 \item If\marginpar{237} you need higher-powered project-management tools, you are using them already.
 \item But\marginpar{238} be sure to focus more on the task than the software.
 \item Always review and \emph{purge} your digital data, otherwise you will lose coordinated orientation of your data.
\end{itemize}

\paragraph{How Do I Apply All This in My World?}

\begin{itemize}
 \item After you have updated your next actions list and your Projects List, give yourself 1--3 hours of ``vertical thinking'' time.
 \item For each project, focus on it top-to-bottom and ask yourself: ``What about this do I want to know, capture, or remember?''
 \item Become\marginpar{239} comfortable with having and using your ideas and focusing your energy constructively on intended outcomes and open loops before you have to.
 \item ``Let our advance worrying become our advance thinking and planning.''--Winston Churchill
\end{itemize}

\section{The Power of the Key Principles}

\subsection{The Power of the Capturing Habit}

\begin{itemize}
 \item By\marginpar{243} implementing the \textit{Getting Things Done} system you will keep your mind distraction-free and ensure a high level of efficiency and effectiveness in your work.
 \item Over time people will trust you more and you will be more self-confident.
 \item This is the power of capturing placeholders for anything that is incomplete or unprocessed in your life.
\end{itemize}

\paragraph{The Personal Benefit}

\begin{itemize}
 \item The\marginpar{244} collection of all outstanding open loops produces anxiety and guilt, but also release and relief.
\end{itemize}

\subparagraph{The Source of the Negative Feelings}

\begin{itemize}
 \item The anxiety and overwhelmingness do not stem from too much to do in too little time, but from the broken implicit agreements you made with yourself when you decided to do those tasks---they\marginpar{245} are symptoms of disintegrated self-trust.
\end{itemize}

\subparagraph{How Do You Prevent Broken Agreements with Yourself?}

\begin{itemize}
 \item Don't make the agreement.
 \begin{itemize}
  \item Means lowering your standards.
  \item Your values make it easier to make choices, but they also add things to make choices on.
  \item Once\marginpar{246} you understand the consequences, you'll say ``no'' more---having a complete and current inventory of all your commitments makes it easier.
  \item If you don't know about them it's much easier to be careless with your commitments.
 \end{itemize}
 \item Complete the agreement.
 \begin{itemize}
  \item The\marginpar{247} two-minute rule helps you feel like you are accomplishing things.
  \item We love doing things as long as we feel that we have completed something.
  \item Professional development $=$ the better you get, the better you'd better get.
 \end{itemize}
 \item Renegotiate the agreement.
 \begin{itemize}
  \item A renegotiated agreement is not a broken one.
  \item Getting\marginpar{248} all your stuff out of your head and into the system is you renegotiating those agreements.
  \item You evaluate them, think about them, and then either do them or set up reminders to do them later.
  \item You can't renegotiate agreements you forgot about, but your brain is still holding you liable for them.
  \item Your psyche knows only ``now,'' so keeping track of multiple things at once automatically fails and generates stress, because you can't do multiple things at once.
  \item To\marginpar{249} your brain it's all just agreements, equally important, kept or broken, and they are broken if you are not doing it right this moment.
 \end{itemize}
\end{itemize}

\subparagraph{The Radical Departure from Traditional Time Management}

\begin{itemize}
 \item Unlike in traditional time management systems, in \textit{Getting Things Done} there is no difference between important and not important agreements---you have to track them all.
 \item Since your psyche cannot distinguish priorities your ``external brain'' must and it must be good (complete, detailed) enough to replace your brain.
 \item Anything kept in your mind will occupy either too much or too little attention.
\end{itemize}

\subparagraph{How Much Capturing Is Required?}

\begin{itemize}
 \item Capturing\marginpar{250} anything will make you feel better, but that's nothing compared to the feeling when you capture everything.
 \item Once there are no reminders on your mind, once there is only one thought, you are in the zone and you have captured everything.
 \item Use your mind to think \emph{about} things rather than \emph{of} things.
 \item Thinking of things doesn't add value to the things, it just adds stress---it's pointless thinking.
 \item Change your habit to capture all (new and old) and stop reminding yourself that things exist.
\end{itemize}

\paragraph{When Relationships and Organizations Have the Capture Habit}

\begin{itemize}
 \item Fear\marginpar{251} of communication gaps in companies result in babysitting and hand-holding.
 \item Any request or information must get into your system to be processed and organized soon and be available for review as an option for action.
 \item Often the weakest link is a person's dulled responsiveness to communications in the system.
 \item In-trays\marginpar{252}, when processed regularly, provide a place for message exchange without interruption.
 \item Real knowledge work requires constant renegotiation so that you feel okay about what you are not doing.
\end{itemize}

\subsection{The Power of the Next-Action Decision}

\begin{itemize}
 \item When\marginpar{253} you evaluate everything for next action required you will get an automatic increase in energy, productivity, clarity, and focus.
 \item Determining\marginpar{254} next actions when things show up vs.\ when they blow up makes a world of difference.
\end{itemize}

\paragraph{Creating the Option of Doing}

\begin{itemize}
 \item Making\marginpar{255} next action decisions before we have to is a learned practice, a conscious application of knowledge-work athletics.
 \item Determining the next action is a prerequisite to getting things done: Having a list of projects on your mind without knowing the next action for each of them means they are not getting done.
 \item The thinking required is simple and short, yet we resist doing it.
 \item Without\marginpar{256} the thinking you won't know what you have to do and many moments which could be filled by doing many actions will pass you by.
 \item When faced with a little bit of time it is too difficult to consider all your projects and choosing the one that fits your current situation.
 \item However, if you've done this thinking earlier it's just a matter of doing the next action---no thinking, no feeling of being overwhelmed.
 \item Define what real doing looks like on the most basic level and organize placeholder reminders that you can trust.
 \item Without the next action the gap between your current reality and what you want to achieve is potentially infinite.
\end{itemize}

\paragraph{Why Bright People Procrastinate the Most}

\begin{itemize}
 \item Most\marginpar{259} creative, intelligent people procrastinate because they imagine failure or not completing the task successfully too vividly.
\end{itemize}

\subparagraph{Intelligent Dumbing Down}

\begin{itemize}
 \item Intelligently dumb down your brain by figuring out the next action.
 \item Defining the next action relieves pressure about any commitment you have.
 \item Nothing changes in the world, but your mind now perceives a doable task; energy goes up and anxiety diminishes.
 \item Anything\marginpar{260} you have to do you are either attracted to doing to repelled by it.
 \item The existence of the next action is usually what makes the difference.
 \item Thinking and deciding takes energy---if you have to do a lot of thinking/deciding you'll feel more tired and overwhelmed.
 \item It's not the contents of your to-do that repel you, it's the amount of thinking required.
 \item Never let your next action lists grow into lists of tasks or subprojects---they have to always remain just lists of discrete, physical next actions.
 \item By avoiding the thinking about the next actions you are not saving any\-thing---that thinking will have to be done no matter what.
 \item It is always better to do the thinking in an organized way when new stuff comes up rather than when you are forced to.
 \item Group\marginpar{261} next actions as they come up into contexts that you can then knock out when you are in said context.
\end{itemize}

\paragraph{The Value of a Next-Action Decision-Making Standard}

\begin{itemize}
 \item Asking ``What is the next action?'' forces clarity, accountability, productivity, and empowerment.
\end{itemize}

\subparagraph{Clarity}

\begin{itemize}
 \item 20\marginpar{262} minutes before the end of the meeting ask the question---it forces more relevant levels of thinking.
\end{itemize}
 
\subparagraph{Accountability}

\begin{itemize}
 \item Asking the question forces the assignment of resources.
\end{itemize}

\subparagraph{Productivity}

\begin{itemize}
 \item Increase\marginpar{263} your operational responsiveness by clarifying actions on the front-end instead of the back-end.
 \item Long-term projects do not belong on the Someday/Maybe List; they just have more action steps until they are done.
\end{itemize}

\subparagraph{Empowerment}

\begin{itemize}
 \item Constant\marginpar{264} fire-fighting prevents a sense of winning.
 \item Getting things going of your own accord, without being forced by external circumstances, increases your self-worth.
 \item Asking the question undermines the victim mentality.
 \item Complaining $=$ unwillingness to move on a changeable situation or non-acceptance of immutable circumstances.
\end{itemize}

\subsection{The Power of Outcome Focusing}

\paragraph{Focus and the Fast Track}

\begin{itemize}
 \item The\marginpar{267} \textit{Getting Things Done} system relieves ``drag'' on your productivity and self-improvement.
\end{itemize}

\paragraph{The Significance of Applied Outcome Thinking}

\begin{itemize}
 \item Clarity\marginpar{268}, productivity, accountability, and empowerment all happen also when you identify the real results you want and the projects needed to achieve them.
 \item Everything you experience as incomplete must have a reference point for complete.
 \item The\marginpar{269} \textit{Getting Things Done} system makes explicit what we all do implicitly and leverages these principles consciously.
 \item Your actions can either be the less-than-conscious responses to the environment or conscious results of your directed focus---that choice is always yours.
 \item The challenge is defining what ``done'' means and what ``doing'' looks like.
 \item First make it up, then make it happen.
\end{itemize}

\paragraph{The Power of Natural Planning}

\begin{itemize}
 \item Natural\marginpar{271} project planning provides an integrated, flexible, aligned way to think through any situation.
 \item Challenge the purpose of everything that you do.
 \item Have ideas, good and bad, and capture them without judgment.
 \item Hone multiple ideas into components, sequences, and priorities aimed toward a specific outcome.
 \item Decide on and take real next actions.
 \item Bring all these components together with appropriate timing and balance.
 \item The natural planning model is natural, but it is not automatic.
 \item Determining\marginpar{272} outcomes and actions for everything must become the norm of your day-to-day life.
\end{itemize}

\paragraph{Shifting to a Positive Organizational Culture}

\begin{itemize}
 \item Even just a few people implementing outcome/action thinking in a group boosts the group's productivity.
 \item ``Why are we doing this?'' and ``What will it look like when it's done successfully?'', with their answers applied to the day-to-day operational level, will have profound results.
 \item Apply\marginpar{273} rigorous questioning of purpose and desired outcomes for meetings and e-mails.
\end{itemize}

\subsection{GTD and Cognitive Science}

\paragraph{Distributed Cognition: The Value of an External Mind}

\begin{itemize}
 \item Your\marginpar{277} mind is designed to have ideas, based upon pattern re\-cog\-ni\-tion/de\-tec\-tion, but it isn't designed to remember much of anything.
\end{itemize}

\paragraph{Relieving the Cognitive Load of Incompletions}

\begin{itemize}
 \item Completion\marginpar{278} of tasks is not necessary to relieve their burden on the mind---a trusted plan ensuring that forward progress will happen is enough.
\end{itemize}

\paragraph{Flow Theory}

\begin{itemize}
 \item Requirements\marginpar{279} for ``flow'' (a.k.a.\ the zone, or ``mind like water''):
 \begin{itemize}
  \item Skills match challenge (too little $=$ anxiety, too much $=$ boredom).
  \item Complete concentration on task at hand.
  \item Limited stimulus field.
  \item Clear goals in mind.
  \item Intrinsic motivation.
  \item Immediate feedback.
 \end{itemize}
 \item You\marginpar{280} can put your attention only on one thing at a time---if that is all that has your attention, you're in flow.
\end{itemize}

\paragraph{Self-Leadership Theory}

\begin{itemize}
 \item Self leadership is the process of controlling your own behavior, influencing yourself through the use of specific behavioral and cognitive strategies.
 \item Behavior-focused\marginpar{281} strategies:
 \begin{itemize}
  \item Raise self-awareness to facilitate behavior management.
  \item Self-observation, self-goal setting, self-reward, self-punishment, and self-caring.
  \item Force yourself to do unpleasant tasks.
 \end{itemize}
 \item Natural reward strategies:
 \begin{itemize}
  \item Motivate or reward yourself with the activity itself.
  \item Focus on the inherently rewarding aspects of unpleasant activities.
 \end{itemize}
 \item Constructive thought pattern strategies:
 \begin{itemize}
  \item Create ways of thinking that positively impact performance.
  \item Self-talk, mental imagery, and replacing dysfunctional beliefs and assumptions.
 \end{itemize}
 \item Providing yourself the right clues, which you will notice at the right time, about the right things is a form of self-cuing.
 \item Dumping your mind into an external brain is a form of natural reward.
 \item Changing from victim state (i.e.\ overwhelmed) to an in-control state (i.e.\ next action) is a positive mindset shift.
 \item Self-leadership\marginpar{282} improves self-efficacy, which is connected to job satisfaction and performance.
\end{itemize}

\paragraph{Goal-Striving/Attainment Via Implementation Intentions}

\begin{itemize}
 \item Ensure goal-striving by creating a cause-and-effect link about when goal-relevant actions will be taken.
 \item When you create implementation intentions ahead of time and decide which actions will be carried out in which contexts, the proper behavior is nearly automatically enacted instead of drawing on your limited reserve of willpower.
\end{itemize}

\paragraph{Psychological Capital (PsyCap)}

\begin{itemize}
 \item Consists of four aspects:
 \begin{itemize}
  \item Self-efficacy\marginpar{283} is the confidence to take on challenging tasks.
  \item Optimism are positive attributions about succeeding.
  \item Hope is preserving toward goals.
  \item Resilience is bouncing back to the original state after facing adversity and problems.
 \end{itemize}
 \item Creating\marginpar{284} a complete inventory of your commitments is self-efficacy.
 \item Drawing connections between purposeful and goal-directed efforts and the successful completion of tasks is optimism.
 \item Front-loaded ``doing work to define work'' is hope.
 \item And anecdotally the \textit{Getting Things Done} system helps you be more resilient.
\end{itemize}

\subsection{The Path of GTD Mastery}

\begin{itemize}
 \item The\marginpar{286} \textit{Getting Things Done} system is a life-long practice with multiple levels of mastery.
 \item Mastery is a demonstrated ability to consistently engage in productive behaviors to achieve clarity, stability, and focus when desired/required.
 \item Any change in your world will test your mastery of the system by confronting you with things that are unclear, unstable, and distracting.
 \item ``Mind like water'' doesn't imply the water is undisturbed, but that water engages appropriately with disturbance, instead of fighting against it.
\end{itemize}

\paragraph{The Three Tiers of Mastery}

\begin{enumerate}
 \item Employing\marginpar{287} the fundamentals of managing workflow.
 \item Implementing a more elevated and integrated total life management system.
 \item Leveraging skill to create clear space and get things done for an ever-expansive expression and manifestation.
\end{enumerate}

\paragraph{Mastering the Basics}

\begin{itemize}
 \item Hour-by-hour, day-to-day focus.
 \item Can\marginpar{288} take a while, because the moves aren't familiar or comfortable.
 \item Becoming sensitized to the need to externalize everything as well as building a habit of actually doing it is a challenge.
\end{itemize}

\subparagraph{It's easy to get off track \ldots}

\begin{itemize}
 \item Easy traps:
 \begin{itemize}
  \item Not capturing everything.
  \item Avoiding\marginpar{289} next action decision making.
  \item Not tracking everything that belongs in Waiting For.
  \item Not keeping the filing systems tidy.
  \item Not treating the Calendar as hard landscape only.
  \item Not doing Weekly Reviews.
 \end{itemize}
 \item When confronted with busy day-to-day lives, if the \textit{Getting Things Done} practices aren't second nature it's easy to get blown off course.
 \item It is easy to avoid next action thinking for things that aren't in crisis mode right now.
\end{itemize}

\subparagraph{\ldots and easy to get back on}

\begin{itemize}
 \item Simply\marginpar{290} revisit the basics:
 \begin{itemize}
  \item Empty your head again.
  \item Clean up your lists.
  \item Add new projects and actions.
  \item Clean what leaked outside of your system.
 \end{itemize}
 \item Getting off track and getting back on happens to everyone.
 \item It may take up to two years to master the basics (i.e.\ have the system fully integrated into your life).
 \item Even just a few steps of \textit{Getting Things Done} help when they are applied, but they work best together as part of a system.
\end{itemize}

\paragraph{Graduate Level---Integrated Life Management}

\begin{itemize}
 \item Week-to-week, month-to-month focus.
 \item Requires subtler level of awareness and practice.
 \item Less\marginpar{291} focus on the system itself and more on utilizing it in more flexible, automated ways.
 \item This level is concerned with issues that are driving the underlying actions, e-mails, meetings, etc.\ (i.e.\ projects, problems, areas of focus).
 \item Mastery of the basics provides ability and room to address a higher level of control and focus.
 \item Hallmarks:
 \begin{itemize}
  \item Complete, current, and clear inventory of projects.
  \item Working map of your roles, accountabilities, etc.
  \item An integrated total life management system.
  \item Challenges and surprises trigger your use of the \textit{Getting Things Done} system rather than throwing you out of it.
 \end{itemize}
\end{itemize}

\subparagraph{When Projects Become the Heartbeat of Your Operational System}

\begin{itemize}
 \item Further\marginpar{292} down the road the Projects List becomes the driver, rather than the reflection, of your next action lists.
 \item Your projects become a truer reflection of your roles and interests.
 \item Very few practitioners or \textit{Getting Things Done} have a complete inventory of their projects objectively and regularly reviewed---those who do make \emph{that} the principle list from which they navigate.
 \item \textit{Getting Things Done} mastery at this level recognizes anything that has your attention (including more subtle desired outcomes) and translates them into achievable outcomes with concrete next actions.
 \item Most people resist acknowledging issues and opportunities until they know they can handle them successfully, not realizing that looking into, exploring, or putting something to bed as impossible are projects as well.
\end{itemize}

\subparagraph{Assessing and Populating Your Projects List from Your Areas of Focus}

\begin{itemize}
 \item All\marginpar{293} projects stem from your roles/accountabilities or an area of interest/engagement.
\end{itemize}

\subparagraph{An Integrated Total Life-Management System}

\begin{itemize}
 \item The System becomes a ``control room'' when all components work together.
 \item You reach \emph{functional awareness} of \textit{Getting Things Done}---you understand the essence and recognize the value of its various parts and can tailor them and how they are implemented to best serve your needs.
 \item You\marginpar{294} can create a placeholder for any kind of unexpected data.
\end{itemize}

\subparagraph{Pressure Produces Greater Rather Than Reduced Utilization of These Practices}

\begin{itemize}
 \item When\marginpar{295} faced with a tense situation you do a mind sweep, identify desired outcomes, projects, and next actions and you do an unplanned Weekly Review in the middle of the week to gain elevated focus.
\end{itemize}

\paragraph{Postgraduate: Focus, Direction, and Creativity}

\begin{itemize}
 \item Using clear internal space to optimize your experience involves two key aspects:
 \begin{enumerate}
  \item Utilizing your free-up focus for exploring the more elevated aspects of your commitments and values.
  \item Leveraging your external mind to produce novel value.
 \end{enumerate}
\end{itemize}

\subparagraph{Freedom to Engage in the Most Meaningful Things}

\begin{itemize}
 \item You will be able to throw whatever crazy idea you have into your own in-tray and trust yourself to handle it properly.
 \item The\marginpar{296} ability to put your attention on the more subtle and elevated levels of your life and work depends on your ability to ``put to bed'' the more operational and mundane aspects that can easily distract and exhaust your creative focus.
 \item Handling the day-to-day won't make you think productively about long-term things, but it makes it much easier.
\end{itemize}

\subparagraph{Leveraging Your External Mind}

\begin{itemize}
 \item Your\marginpar{297} focus shifts from implementing the most effective way of dealing with the inputs and demands to optimally taking advantage of self-created contexts and triggers to produce creative ideas that would normally not occur.
 \item The Weekly Review produces productive reflection, connecting dots, discovering new things, etc.---what other aspects of your experience and relationships could be enhanced with the same kind of triggers for reflection?
 \item When\marginpar{298} freed from the need to remember, the mind is a great at creative thinking about what is in front of it.
 \item You are randomly in your ``smarts''---build systems and processes to take advantage of them and trigger them as often as possible.
\end{itemize}

\begin{itemize}
 \item These three levels are not sequential---you can display parts of each from the beginning---but there are no shortcuts.
 \item You\marginpar{299} are continually involved with all of these levels---mastery of \textit{Getting Things Done} will simply reflect the elegant equanimity with which you are engaged with all of them.
\end{itemize}

\subsection{Conclusion}

\begin{itemize}
 \item Get\marginpar{300} your personal physical hardware set up.
 \item Get your workstation organized.
 \item Get in-trays.
 \item Create\marginpar{301} a workable and easily accessed personal reference system---for work and for home.
 \item Get a good list-management organizer that you are inspired to play with.
 \item Make any changes to your working environment you have been contemplating.
 \item Gather everything into your system.
 \item Share anything of value you've learned with someone else.
 \item Review \textit{Getting Things Done} in 3--6 months.
\end{itemize}

\end{document}
